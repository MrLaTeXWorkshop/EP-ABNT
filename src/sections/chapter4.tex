\chapter{Medidas de tendência central e de variabilidade}

    Tendência central ou centralidade, refere-se a propensão de dados quantitativos acumularem
    em proximidade de um valor central. Em outras palavras, a partir destas medidas é possível 
    descobrir um número que ocupa a posição central em um conjunto de valores.

    Abaixo pode ser encontrado uma tabela feita para parear os dados quantitativos do banco de dados, 
    e encontrar as medidas de tendência central. Ela também apresenta o desvio padrão, importante 
    parâmetro estátistico para encontrar o grau de variação de um conjunto de elementos.

    \begin{table}[htb]
      \caption{Medidas de tendência central e de variabilidade}
      \centering
        \begin{tabular}{ |P{2.8cm} |P{2cm}|P{3cm}|P{3cm}|P{3cm}|  }
          \hline
            &        Idade & Pressão Sanguínea em repousou(kPa) & Colesterol  (mg/dl) & Pico de frequência cardíaca(bpm)\\
          \hline
          Média & 54.36 & 131.62 & 246.26 & 149.64 \\
          \hline
          Mediana & 55 & 130 & 240 & 153 \\
          \hline
          Moda & 58 & 120 & 204, 234, 197 & 162 \\
          \hline
          Desvio Padrão & 9.08 & 17.53 & 51.83 & 22.90 \\
          \hline
        \end{tabular}
      \legend{Fonte: Produzido pelos próprios autores}
    \end{table}

