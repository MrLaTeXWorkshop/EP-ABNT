\chapter{Medidas de tendência central e de variabilidade}

    Tendência central ou centralidade, refere-se a propensão de dados quantitativos acumularem
    em proximidade de um valor central. Em outras palavras, a partir destas medidas é possível 
    descobrir um número que ocupa a posição central em um conjunto de valores.

    Eis aqui as medidas de tendência central mais utilizadas e que também foram utilizadas nesse 
    estudo:

    \begin{enumerate}[label={\alph*)}]
      \addtolength{\itemindent}{1.25cm}
      \item Média: Também chamado de média aritmética:, é a soma de todos os elementos de um conjunto;
      \item Mediana: Valor em um conjunto de dados que divide o grupo ao meio;
      \item Moda: Valor que ocorre com a maior frequência em um grupo de dados;
    \end{enumerate}

    Abaixo pode ser encontrado uma tabela feita para parear os dados quantitativos do banco de dados, 
    e encontrar as medidas de tendência central. Ela também apresenta o desvio padrão, importante 
    parâmetro estatístico para encontrar o grau de variação de um conjunto de elementos.

    \begin{table}[htb]
      \caption{Medidas de tendência central e de variabilidade}
      \centering
        \begin{tabular}{ |P{2.8cm} |P{2cm}|P{3cm}|P{3cm}|P{3cm}|  }
          \hline
            & Idade & Pressão Sanguínea em repousou(kPa) & Colesterol  (mg/dl) & Pico de frequência cardíaca(bpm)\\
          \hline
          Média & 54.36 & 131.62 & 246.26 & 149.64 \\
          \hline
          Mediana & 55 & 130 & 240 & 153 \\
          \hline
          Moda & 58 & 120 & 204, 234, 197 & 162 \\
          \hline
          Desvio Padrão & 9.08 & 17.53 & 51.83 & 22.90 \\
          \hline
        \end{tabular}
      \legend{Fonte: Produzido pelos próprios autores}
    \end{table}

    Analisando os dados, o primeiro detalhe a ser observado é a idade, onde é percebível que em média
    as pessoas apresentam idade quase avançada(perto dos 65 anos), além disso, o desvio padrão indica 
    uma baixa dispersão(9\%), logo indicando que existe pouca variação entre as idades 
    (dados homogêneos).

    Sobre a pressão sanguínea em repouso, os números parecem estar normais, possuindo valores 
    que são considerados saudáveis. Enquanto isso, desta vez o desvio padrão demonstra que neste caso 
    os dados possuem uma dispersão media.

    O nível total de colesterol já demonstra um quadro um pouco mais preocupante, todas as medidas 
    de tendência central demonstra dados que estão muito altos em comparação com os dados de referência.
    Desta vez, os dados se apresentam heterogêneos, devido a um número alto do desvio padrão.

    Por fim, temos o pico da frequência cardiáca. Neste caso, temos uma situação interessante, pois 
    a média e mediana se apresentam apenas um pouco acima do que a referência indica, porém o número 
    mais frequente nestes dados(também conhecido como a moda), é de 162 bpm, o que é um número relativamente 
    alto. Assim como o colesterol, o desvio padrão desses dados sugere uma dispersão alta.

    \nocite{sobremedidascentrais}
    \nocite{colesterol}
    \nocite{frequenciacardiaca}
