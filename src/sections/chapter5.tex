\chapter{Intervalo de confiança para proporção de interesse}

    De uma base amostral contendo 303 indivíduos, para se aprofundar mais nas características deste grupo 
    é necessário realizar testes mais específicos. Com isso serão realizados testes para se descobrir os 
    intervalos de confiança em determinadas situações. Existem duas formas de testes. 

    \section{Proporção Populacional de Interesse}

    \begin{flalign}
      IC(1-\alpha)\% &= \hat{P} \pm Z_\frac{\alpha}{2} * \sqrt{\frac{\hat{P}*(1-\hat{P})}{n}} \\\nonumber
      IC(95)\% &= 0,6832 \pm 1,96 * \sqrt{\frac{0,6832*(0,3168)}{303}} \\\nonumber
      IC(95)\% &= 0,6832 \pm 0,0524 \\\nonumber
      IC(95)\% &= [0,6308 ; 0,7356] &&\nonumber 
    \end{flalign}   

    Com os resultados mostrados, é possível observar que a chance de um homem ter um infarto é de 
    63,08\% a 73,56\% com intervalo de confiança de 95\%.

    \begin{flalign}
      IC(1-\alpha)\% &= \hat{P} \pm Z_\frac{\alpha}{2} * \sqrt{\frac{\hat{P}*(1-\hat{P})}{n}} \\\nonumber
      IC(95)\% &= 0,3267 \pm 1,96 * \sqrt{\frac{0,3267*(0,6733)}{303}} \\\nonumber
      IC(95)\% &= 0,3267 \pm 0,0269 \\\nonumber
      IC(95)\% &= [0,2998 ; 0,3536] &&\nonumber 
    \end{flalign}  

    Com os resultados mostrados, é possível observar que  a chance de uma pessoa apresentar angina 
    após a execução de atividades físicas é de 29,98\% a 35,36\% com intervalo de confiança de 95\%.
 
    \section{Média Populacional de Interesse}

    \begin{flalign}
      IC(1-\alpha)\% &= \bar{x} \pm Z*\frac{\alpha}{2} * \frac{\sigma}{\sqrt{N}} \\\nonumber
      IC(95)\% &= 60,45 \pm 1,96 * \frac{5.1}{\sqrt{183}} \\\nonumber
      IC(95)\% &= 60,45 \pm 0,74 \\\nonumber
      IC(95)\% &= [59,1 ; 61,19] &&\nonumber
    \end{flalign}