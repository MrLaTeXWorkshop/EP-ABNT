\chapter{Intervalo de confiança para proporção de interesse}

    De uma base amostral contendo 303 pessoas, foram selecionadas todas com idade
    igual ou superior a 53 anos, totalizando 183 pessoas. Desta forma, foi realizado 
    o intervalo de confiança por proporção.

    \begin{flalign}
      IC(1-\alpha)\% &= \hat{P} \pm Z_\frac{\alpha}{2} * \sqrt{\frac{\hat{P}*(1-\hat{P})}{n}} \\\nonumber
      IC(95)\% &= 0.604 \pm 1.96 * \sqrt{\frac{0.604*(0.396)}{303}} \\\nonumber
      IC(95)\% &= 0.604 \pm 0.055 \\\nonumber
      IC(95)\% &= [0.549 ; 0.659] &&\nonumber 
    \end{flalign}   

    Após analisar o resultado, é possível observar que a chance dos ataques cardivasculares ocorrerem em pessoas acima 
    dos 53 anos, vária de um intervalo de 54,9\% à 65,9\% com intervalo de confiança equivalente a 95\% 

    Outro teste realizado no para a amostra foi o intervalo de confiança para média populacional. Porém,
    como o desvio padrão populacional desconhecido. 

    \begin{flalign}
      IC(1-\alpha)\% &= \bar{x} \pm Z*\frac{\alpha}{2} * \frac{\sigma}{\sqrt{N}} \\\nonumber
      IC(95)\% &= 60.45 \pm 1.96 * \frac{5.1}{\sqrt{183}} \\\nonumber
      IC(95)\% &= 60.45 \pm 0.74 \\\nonumber
      IC(95)\% &= [59.1 ; 61.19] &&\nonumber
    \end{flalign}

