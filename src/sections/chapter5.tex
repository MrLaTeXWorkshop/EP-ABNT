\chapter{Intervalo de confiança para proporção de interesse}

    De uma base amostral contendo 303 pessoas, foram selecionadas todas com idade
    igual ou superior a 53 anos, totalizando 183 pessoas. Desta forma, foi realizado 
    o intervalo de confiança por proporção.
    
    \begin{equation} \label{eq1}
    \begin{split}
    & IC(1-\alpha)\%= \hat{P} \pm Z_\frac{\alpha}{2} * \sqrt{\frac{\hat{P}*(1-\hat{P})}{n}} \\
    & IC(95)\% = 0,604 \pm 1,96 * \sqrt{\frac{0,604*(0,396)}{303}} \\
    & IC(95)\% = 0,604 \pm 0,055 \\
    & IC(95)\% = [0,549 ; 0,659] 
    \end{split}
    \end{equation}
    
    \begin{equation} \label{eq2}
    \begin{split}
    & IC(1-\alpha)\%= \bar{x} \pm \tau_\frac{\alpha}{2};n-1 \frac{S}{\sqrt{n}} \\
    & IC(95)\% = 60,45 \pm 1,960 * \frac{5,10}{\sqrt{183}} \\
    & IC(95)\% = 60,45 \pm 0,74 \\
    & IC(95)\% = [59,1 ; 61,19] 
    \end{split}
    \end{equation}



