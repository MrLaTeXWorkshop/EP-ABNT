\chapter{Classificação das variáveis}

    \setcounter{section}{0}

    Antes de começar a aprofundar no estudo, inicialmente deve se fazer uma análise dos elementos 
    presentes no conjunto. Ao selecionar uma amostragem, são analisadas informações capazes de explicar 
    e de mostrar as características da população em questão.

    Essas características são denominadas de variáveis que podem ser classificadas de diferentes formas.

    \section{Qualitativas Nominais}

    Variáveis de características não numérica, que nomeia ou rótulo as características por meio de números ou símbolos. 
    
    Na amostragem em questão, as variáveis a seguir são classificadas dessa forma: 

    \section{Qualitativas Ordinais}

    Variáveis de características não numérica, que mantém uma relação de ordem. 
  
    Na amostragem em questão, as variáveis a seguir são classificadas dessa forma: 

    \section{Quantitativas Discretas}

    Variáveis que assumem valores inteiros e pontuais pertencentes a um conjunto enumerável. 

    Na amostragem em questão, as variáveis a seguir são classificadas dessa forma: 

    \section{Quantitativas Contínuas}

    Variáveis que assumem valores qualquer valor real em um intervalo, associados a medição. 

    Na amostragem em questão, as variáveis a seguir são classificadas dessa forma: 
