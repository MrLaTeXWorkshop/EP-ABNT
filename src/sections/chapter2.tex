\chapter{Classificação das variáveis}

    \setcounter{section}{0}

    Antes de começar a aprofundar no estudo, inicialmente deve se fazer uma análise dos elementos 
    presentes no conjunto. Ao selecionar uma amostragem, são analisadas informações capazes de explicar 
    e de mostrar as características da população em questão.

    Essas características são denominadas de variáveis que podem ser classificadas de diferentes formas.

    \section{Qualitativas Nominais}

    Variáveis de características não numérica, que nomeia ou rótula as características por meio de números ou símbolos. 
    
    Na amostragem em questão, as variáveis a seguir são classificadas dessa forma: 

    \begin{enumerate}[label={\alph*)}]
        \addtolength{\itemindent}{1.25cm}
        \item "sex" = gênero;
        \item "fbs" = exercício induziu angina ;
        \item "exng" = açúcar no sangue em jejum acima de 120 mg/dl;
        \item "oldpeak" = depressão de ST induzida por exercício em relação ao repouso;
        \item "cp" = tipo de dor no peito;
        \item "restecg" = resultados eletrocardiográficos em repouso
        ;
    \end{enumerate}

    \section{Qualitativas Ordinais}

    Variáveis de características não numérica, que mantém uma relação de ordem. 
  
    Na amostragem em questão, as variáveis a seguir são classificadas dessa forma: 

    “Não apresenta variáveis nessa amostragem com essa classificação”

    \section{Quantitativas Discretas}

    Variáveis que assumem valores inteiros e pontuais pertencentes a um conjunto enumerável. 

    Na amostragem em questão, as variáveis a seguir são classificadas dessa forma: 

    \begin{enumerate}[label={\alph*)}]
        \addtolength{\itemindent}{1.25cm}
        \item "age" = idade;
        \item "thalachh" = frequência cardíaca máxima alcançada;
    \end{enumerate}

    \section{Quantitativas Contínuas}

    Variáveis que assumem valores qualquer valor real em um intervalo, associados a medição. 

    Na amostragem em questão, as variáveis a seguir são classificadas dessa forma: 

    \begin{enumerate}[label={\alph*)}]
        \addtolength{\itemindent}{1.25cm}
        \item "trtbps" = pressão arterial em repouso em mm Hg;
        \item "chol" = colesterol em mg/dl;
    \end{enumerate}

    \nocite{classificacao}