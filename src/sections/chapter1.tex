\chapter*[Introdução]{Introdução}

    \addcontentsline{toc}{chapter}{Introdução}

    As doenças cardiovasculares é um conjunto de doenças do coração e dos vasos sanguíneos, incluindo 
    problemas estruturais e coágulos. De acordo com dados distribuídos pela Organização Mundial de Saúde(OMS), é estimado que no ano de 2016, 
    17.9 milhões de pessoas morreram por conta de doenças cardiovasculares, representando 31\% 
    de todas as mortes em nível global. Além disso, de acordo com a Sociedade Brasileira de Cardiologia, doenças cardiovasculares(DCV), tem sido a principal causa 
    de mortalidade no Brasil desde a década de 1960.

    Devido à pandemia ocasionada pelo COVID-19, admite-se que muitos desses casos vão ocorrer
    com mais frequência, principalmente em pessoas mais velhas devido ao estresse. Uma matéria da 
    CNN Brasil de Janeiro deste ano, comenta dados de uma pesquisa feita no Brasil, afirma 
    que: "o número de mortes por doenças cardiovasculares cresceu até 132\% no Brasil durante a pandemia"
    \cite{abccardiol}.

    Sendo assim, por este ser um assunto relevante no contexto atual, foi selecionado um banco de 
    dados, fornecido pelo site Kaggle, uma subsidiária da Google LLC, com fôco em Cientistas 
    de Dados e Machine Learning, afim de estudar estátisticamente as váriaveis presentes na amostra.

    \newpage
    \chapter{Recolhimento dos dados}

    A princípio, para tratar mais a fundo sobre o assunto, foi necessário escolher um banco 
    de dados com dados suficientes para análise. Sendo assim, o site Kaggle foi uma escolha
    rápida e eficiente, oferecendo um espaço amostral de 303 pessoas. 

    \nocite{bancodedados}
    \nocite{oms}

    \newpage