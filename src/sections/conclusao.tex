\phantompart

\chapter*[Conclusão]{Conclusão}
    \addcontentsline{toc}{chapter}{Conclusão}

    Com todos os dados recolhidos, classificados e analisados, podemos começar a fazer nossas suposições 
    a partir do que foi percebível, espera-se que analisando todos os 303 casos sejá possível dar mais relevância aos dados.
    O primeiro dado que se destaca é o colesterol que a partir da análise 
    das medidas de tendência central foi demonstrado que estes estão muitos mais altos do que o número desejável.
    Isso pode levar a conclusão que a primeira coisa a se fazer para diminuir as chances de um ataque cardíaco seja 
    a partir da redução do nível total do colesterol.

    Uma coisa bem importante que ainda não foi mencionada, é de que há um numero desproporcionalmente maior 
    de homens com colesterol alto do que mulheres com colesterol alto. Todavia, dos 303 dados analisados, 210 são
    dados de pessoas do sexo masculino, logo, necessitando de mais dados para poder tirar quaisquer conclusões significativas.

    Por fim, gostariamos também de citar mais uma coisa importante: as amostras coletadas apresentam um total 
    de 14 variáveis e que destas, apenas 10 foram utilizadas e sem contar que nesse grupo de variáveis utilizadas, houve
    aquelas que não foram aprofundadas. O motivo disso se deve ao fato que o objetivo desse relatório é 
    não fazer muitas extrapolações em certas explicações e para facilitar o agrupamento.
