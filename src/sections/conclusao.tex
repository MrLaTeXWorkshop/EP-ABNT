\phantompart

\chapter*[Conclusão]{Conclusão}
    \addcontentsline{toc}{chapter}{Conclusão}

    Após trabalhar nessa linguagem por quase um mês, ficou claro para toda a equipe que Haskell é uma linguagem diferenciada. Possui
    uma história que foi essencial para sua formação, sem esquecer, é claro, que foi uma lingua desenvolvida de forma comunitária.
    Há grandes obstáculos em ingressar nessa lingua, principalmente por documentação excassa e a documentação oficial em maioria ser
    paga ou extremamente complexa para iniciantes. 

    Por muito tempo, Haskell não foi uma lingua unificada, cada pessoa que entrava no projeto criava uma versão diferente sem que um líder
    principal coordenasse como ela estava evoluindo. Falta de uma unidade atrasou um pouco a lingua ser adotada pela comunidade.
    Por conta da falta de uma documentação única, a dificuldade de aprendizado foi outro grande ponto negativo que impactou diretamente
    na falta de profissionais que a utilização, ficando apenas fechada em um ambiente científico como universidades.

    Entretanto depois de tudo que vimos, a linguagem se apresenta de forma bem mais positiva do que todas essas ideias
    citadas acima. Ela apresenta vários recursivos interessantes, como cálculos lambda, recursões simples e amarrações de funções
    em variáveis de forma simples. Há vários tutoriais distribuidos pela internet. Mesmo que
    poucas pessoas programem nessa lingua, possui uma comunidade bem forte que a mantém.

    Por fim, o grupo entendeu que Haskell desempenhou seu papel na história da computação. Além disso,
    várias empresas ainda o adotam por entenderem a importância de seu uso, feito para cálculos científicos e precisos.
    Ainda dentro do contexto de programação funcional e uso de cálculos lambda, Haskell ainda é uma, se não a melhor,
    lingua para ser utilizado.