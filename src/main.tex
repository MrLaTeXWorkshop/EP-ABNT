\documentclass[
  % -- opções da classe memoir --
  12pt,				         % tamanho da fonte
  oneside,			       % para impressão apenas no recto. Oposto a twoside
  a4paper,			       % tamanho do papel. 
  article,
  % -- opções da classe abntex2 --
  %chapter=TITLE,		   % títulos de capítulos convertidos em letras maiúsculas
  %section=TITLE,		   % títulos de seções convertidos em letras maiúsculas
  %subsection=TITLE,	 % títulos de subseções convertidos em letras maiúsculas
  %subsubsection=TITLE % títulos de subsubseções convertidos em letras maiúsculas
  % -- opções do pacote babel --
  english,		       	 % idioma adicional para hifenização
  brazil,			      	 % o último idioma é o principal do documento
]{abntex2}
\usepackage{meupacote}
  

\pgfplotsset{compat=1.8}
\usepgfplotslibrary{statistics}
% ---
% 
% ---
\graphicspath{{../images/}}
% O tamanho do parágrafo é dado por:

% ---
% Altera as margens padrões
% ---
\setlrmarginsandblock{3cm}{3cm}{*}
\setulmarginsandblock{3cm}{3cm}{*}
\checkandfixthelayout
% ---

\setlength{\parindent}{1.3cm}
% ---
% 
% ---
% Controle do espaçamento entre um parágrafo e outro:
\setlength{\parskip}{0.2cm}  % tente também \onelineskip
% ---
% 
% ---
% Espaçamento simples
\SingleSpacing
% ---
% Informações de dados para CAPA
% ---
\titulo{Um estudo estatístico sobre ataques cardíacos e seu prognóstico}
\autor{Gustavo Lopes \and Henrique Cota \and Homenique Vieira \and Lucas Santiago \and Rafael Amauri \and  Thiago Henriques}
\local{Belo Horizonte}
\data{2021}
\instituicao{%
  Pontifícia Universidade Católica Minas Gerais
  }
\tipotrabalho{Trabalho de Estatística}
% ---
% 
% ---
% informações do PDF
\makeatletter
\hypersetup{
     	%pagebackref=true,
		pdftitle={\@title}, 
		pdfauthor={\@author},
    	pdfsubject={Um estudo estatístico sobre ataques cardíacos e formas de prognosticar},
	    pdfcreator={GL, HV, LS, RF, TH},
		pdfkeywords={abnt}{latex}{abntex}{abntex2}{Estatística e Probabilidade}{Ataques cardíacos}, 
		colorlinks=true,       		% false: boxed links; true: colored links
    	linkcolor=black,          	% color of internal links
    	citecolor=blue,        		% color of links to bibliography
    	filecolor=magenta,      		% color of file links
		urlcolor=blue,
		bookmarksdepth=4
}
\makeatother

\makeindex
% ---
% Iniciando efetivamente o documento
% ---
\begin{document} 

    % Fazer com que as secções sejão subcapitulos
    \renewcommand{\thesection}{\noindent\arabic{chapter}.\arabic{section}}. 
    % ---
    % Selecionando linguagem
    % ---
    \selectlanguage{brazil}
    % ---
    % Retira espaço extra obsoleto entre as frases.
    % ---
    \frenchspacing
    % ---
    % Imprimir a capa 
    % ---
    \imprimircapa
    %
    % --- 
    \begin{abstract}

      "Estátistica Descritiva e Estatística Inferencial" são áreas de grande importância
      para a Estátistica, pois elas ajudão a descrever uma população através de um conjunto de dados 
      amostrais. Este relatório técnico para a disciplina de Estatística e Probabilidade serve 
      como um estudo de tais áreas, fazendo a descrição de dados coletados do site Kaggle sobre 
      o assunto: ataques cardiovasculares e suas predições.
            
      \vspace{\onelineskip}
      
      \noindent

      \textbf{Palavras-chave:} Ataques cardiovasculares, Estatística e Probabilidade, Kaggle,
      Estátistica Descritiva e Estatística Inferencial
      \end{abstract}
    % ---
    %
    %
    \newpage 

    % ---
    % inserir lista de ilustrações
    % ---
    \pdfbookmark[0]{\listfigurename}{lof}
    \listoffigures*
    \cleardoublepage
    % ---

    % ---
    % inserir lista de tabelas
    % ---
    \pdfbookmark[0]{\listtablename}{lot}
    \listoftables*
    \cleardoublepage
    % ---
    % Imprimir a tabela de conteúdos(Sumário)
    % ---
    \pdfbookmark[0]{\contentsname}{toc}
    \tableofcontents*
    \cleardoublepage
    % ---
    % PARTE TEXTUAL
    % ---
    \textual
    % ---
    % Criar nova página e então iniciar a escrita
    % ---
    \newpage
    
    \import{./sections}{chapter1.tex}

    \newpage

    \import{./sections}{chapter2.tex}

    \newpage

    \import{./sections}{chapter3.tex}

    \newpage

    \import{./sections}{chapter4.tex}

    \newpage

    \import{./sections}{chapter5.tex}

    \newpage

    \import{./sections}{conclusao.tex}

    \newpage

    \postextual

    \bibliography{referencias}
    \import{./sections}{appendice.tex}
    

\end{document}